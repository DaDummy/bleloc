\chapter{Bug Report}
\begin{enumerate}
\item {[app] Scan results may get lost when logging out in the wrong moment (state = open)}
\begin{itemize}	
\item \textbf{Summary: } Downloaded scan results may get lost when reinstalling or logging out in the wrong moment.\\
\item \textbf{Steps to reproduce: }
\begin{itemize}
	\item	Mark a device for tracking
	\item	Make sure new tracking results are available for this device
	\item	Visit the dashboard to trigger downloading these new tracking results
	\item	Reinstall or log out of the app without visiting the dashboard again
	\item	If done just right, the downloaded results will be lost forever
\end{itemize}
\item \textbf{What is the current bug behavior? }Freshly downloaded results may be lost.\\
\item \textbf{What is the expected correct behavior? }Scan results are transferred in a safe way that prevents data loss.\\
\item \textbf{Reason? }Scan results are reported as successfully retrieved to the server before they have been backed up to user data sync.\\
\item \textbf{Possible fixes: }Only report scan results to be successfully retrieved once user data sync has backed them up.\\
\end{itemize}

\item {[backend] Database modifications made using an injected EntityManager are not written back to the database (state = closed)}
\begin{itemize}	
	\item \textbf{Summary: }Database modifications made using an injected EntityManager (@PersistenceContext annotation) are not written back to the database. It seems like those queries are executed, but immediately rolled back afterwards.\\
	\item \textbf{Steps to reproduce: }
	\begin{itemize}
	\item Sign up using the app\\
	\item Try to login with the credentials just used to sign up\\
	\end{itemize}
	\item \textbf{What is the current bug behavior? } Login fails.\\
	\item \textbf{What is the expected correct behavior? } Login should be successful.\\
	\item \textbf{Reason? } The user account record has not been persisted to the database.\\
	\item \textbf{Possible fixes: } Refactor JAX-RS resources to be Enterprise Java Beans, so JTA and it's associated user transaction management are properly initialized and committed.\\
\end{itemize}

\item  {[backoffice] Database modifications made using an injected EntityManager are not written back to the database         (state = closed)}
\begin{itemize}	
	\item \textbf{Summary: } Database modifications made using an injected EntityManager (@PersistenceContext annotation) are not written back to the database. It seems like those queries are executed, but immediately rolled back afterwards.\\
	\item \textbf{Steps to reproduce: }
	\begin{itemize}
		\item Login to the backoffice using the initial administrator account
		\item Create a new backoffice account
		\item Open the list of existing backoffice accounts
	\end{itemize}
	\item \textbf{What is the current bug behavior? } The newly created backoffice account is not listed.\\
	\item \textbf{What is the expected correct behavior? } The newly created backoffice account should be listed.\\
	\item \textbf{Reason? } The backoffice account record has not been persisted to the database.\\
	\item \textbf{Possible fixes: } Refactor Servelet resources to be Enterprise Java Beans, so JTA and it's associated user transaction management are properly initialized and committed.\\
\end{itemize}

\item {[server-model] DeviceHashTableStore's implementation requires it to be a Singleton, but it isn't use like one (state = closed)}
\begin{itemize}	
	\item \textbf{Summary: }The way the "store" currently is used does not allow it to store anything. Turn it into a singleton and use CDI to access it.\\
	\item \textbf{Steps to reproduce: }
	\begin{itemize}
		\item Run unit test: DeviceManagementTest::postRegister()\\
	\end{itemize}
	\item \textbf{What is the current bug behavior? } The device is not listed in the DeviceHashTable and thus not sent to the client during filter download\\
	\item \textbf{What is the expected correct behavior? } The device should be listed in the DeviceHashTable\\
	\item \textbf{Reason? } DeviceHashTableStore's implementation and the way it is used are not compatible.\\
	\item \textbf{Possible fixes: } Adjust usage of DeviceHashTableStore to match its implementation. (Have only one instance used globally)\\
\end{itemize}


\item {[backend] Query /backend/user/register returns wrong status code when an empty/invalid user name is supplied (state = closed)}
\begin{itemize}	
	\item \textbf{Summary: }Query /backend/user/register returns wrong status code when an empty/invalid user name is supplied.\\
	\item \textbf{Steps to reproduce: }
	\begin{itemize}
		\item Run unit test: UserAccountManagementTest::postRegisterEmtpyName().\\
	\end{itemize}
	\item \textbf{What is the current bug behavior? }HTTP status returned: 401 Unauthorized.\\
	\item \textbf{What is the expected correct behavior? }HTTP status expected: 400 Bad request.\\
	\item \textbf{Reason? } Wrong status is written to the response object.\\
	\item \textbf{Possible fixes: }Write correct status to the response object..\\
\end{itemize}


\item {[backend] RequireUserAccountFilter aborts with wrong status code when authentication failed (state = closed)}
\begin{itemize}	
	\item \textbf{Summary: } RequireUserAccountFilter aborts with wrong status code when authentication failed.\\
	\item \textbf{Steps to reproduce: }
	\begin{itemize}
		\item Run unit test: RequireUserAccountFilterTest::filterFail()\\
	\end{itemize}
	\item \textbf{What is the current bug behavior? }HTTP status returned: 401 Unauthorized.\\
	\item \textbf{What is the expected correct behavior? }HTTP status expected: 403 Forbidden.\\
	\item \textbf{Reason? }Wrong status is written to the response object.\\
	\item \textbf{Possible fixes: }Write correct status to the response object.\\
\end{itemize}

\item {[backend] Query /backend/user/register returns wrong status code when an already taken user name is supplied (state = closed)}
\begin{itemize}	
	\item \textbf{Summary: }Query /backend/user/register returns wrong status code when an already taken user name is supplied.\\
	\item \textbf{Steps to reproduce: }
	\begin{itemize}
		\item Run unit test: UserAccountManagementTest::postRegisterNameTaken()\\
	\end{itemize}
	\item \textbf{What is the current bug behavior? }HTTP status returned: 401 Unauthorized.\\
	\item \textbf{What is the expected correct behavior? }HTTP status expected: 409 Conflict.\\
	\item \textbf{Reason? }Wrong status is written to the response object.\\
	\item \textbf{Possible fixes: }Write correct status to the response object.\\
\end{itemize}


\item  {[backend] Query /backend/user/login returns wrong status code when login failed (state = closed)}
\begin{itemize}	
	\item \textbf{Summary: }The query /backend/user/login returns wrong status code when login failed.\\
	\item \textbf{Steps to reproduce: }
	\begin{itemize}
		\item Run unit test: UserAccountManagementTest::postLoginBadUser()
	\end{itemize}
	\item \textbf{What is the current bug behavior? }HTTP status returned: 401 Unauthorized.\\
	\item \textbf{What is the expected correct behavior? }HTTP status expected: 403 Forbidden.\\
	\item \textbf{Reason? }Wrong status is written to the response object.\\
	\item \textbf{Possible fixes: }Write correct status to the response object.\\
\end{itemize}


\item {[backend] Query /backend/scan/filter always fails due to NullPointerException (state = closed)}
\begin{itemize}	
	\item \textbf{Summary: }The Query /backend/scan/filter always fails due to NullPointerException.\\
	\item \textbf{Steps to reproduce: }
	\begin{itemize}
		\item Run unit test: DeviceTrackingTest::getFilter().\\
	\end{itemize}
	\item \textbf{What is the current bug behavior? }The Query fails due to NullPointerException.\\
	\item \textbf{What is the expected correct behavior? }The Query should process successfully and return the serialized form of the current scan filter.\\
	\item \textbf{Reason? } Access to the DeviceHashTableStore is not correctly implemented.\\
	\item \textbf{Possible fixes: } Fix access to the DeviceHashTableStore.\\
\end{itemize}

\item {[backend] Query /backend/device/register does not add device to the DeviceHashTable (state = closed)}
\begin{itemize}	
	\item \textbf{Summary: }Query /backend/device/register does not add device to the DeviceHashTable.\\
	\item \textbf{Steps to reproduce: }
	\begin{itemize}
		\item Run unit test: DeviceManagementTest::postRegister()\\
	\end{itemize}
	\item \textbf{What is the current bug behavior? }The device is not added to the DeviceHashTable.\\
	\item \textbf{What is the expected correct behavior? }The device should get added to the DeviceHashTable.\\
	\item \textbf{Reason? }The event regarding the addition of the device is not relayed to the DeviceHashTable manager class.\\
	\item \textbf{Possible fixes: }Fix event propagation/subscription.\\
\end{itemize}

\item {[server-model] UserAccountStore.remove() throws Exception instead of performing a cascading delete (state = closed)}
\begin{itemize}	
	\item \textbf{Summary: }UserAccountStore.remove() throws an Exception instead of performing a cascading delete.\\
	\item \textbf{Steps to reproduce: }
	\begin{itemize}
		\item Run unit test: UserAccountManagementTest::postDeleteOk()\\
	\end{itemize}
	\item \textbf{What is the current bug behavior? }UserAccountStore.remove() throws an Exception.\\
	\item \textbf{What is the expected correct behavior? }UserAccountStore.remove() should run successfully and trigger a cascading delete.\\
	\item \textbf{Reason? }Faulty Annotations regarding foreign key constraint handling.\\
	\item \textbf{Possible fixes: }Place correct Annotations regarding foreign key constraint handling.\\
\end{itemize}


\item {[server-model] DeviceStore::registerDeviceStoreListener(...) throws UnsupportedOperationException (state = closed)}
\begin{itemize}	
	\item \textbf{Summary: }DeviceStore::registerDeviceStoreListener(...) throws UnsupportedOperationException. It should correctly register the listener and send events to it when they occur. (Only events that originate in this very DeviceStore instance are relevant).\\
	\item \textbf{Steps to reproduce: }
	\begin{itemize}
		\item Run unit test: DeviceManagementTest::postRegister()\\
	\end{itemize}
	\item \textbf{What is the current bug behavior? }DeviceStore::registerDeviceStoreListener(...) throws UnsupportedOperationException.\\
	\item \textbf{What is the expected correct behavior? }It should correctly register the listener and send events to it when they occur.\\
	\item \textbf{Reason? }Faulty implementation.\\
	\item \textbf{Possible fixes: }Implement correct behaviour.\\
\end{itemize}


\item {[app] User data sync uploads duplicate "new" entries under certain conditions (state = closed)}
\begin{itemize}	
	\item \textbf{Summary: } User data sync uploads entries that were initially uploaded as new again as duplicate "new" entries on every subsequent sync.\\
	\item \textbf{Steps to reproduce: }
	\begin{itemize}
		\item Add a device locally\\
		\item Restart the app to trigger a sync\\
		\item Restart the app again to trigger another sync\\
	\end{itemize}
	\item \textbf{What is the current bug behavior? }The added device that was already uploaded during the first sync is duplicated and uploaded again on the second sync.\\
	\item \textbf{What is the expected correct behavior? }The added device should only be uploaded once as "new".\\
	\item \textbf{Reason? }Faulty implementation of UserDataSyncManager.\\
	\item \textbf{Possible fixes: }Fix implementation of UserDataSyncManager.\\
\end{itemize}


\item {[app] It's possible to register unpaired devices (state = closed)}
\begin{itemize}	
	\item \textbf{Summary: }It's possible to register unpaired devices.\\
	\item \textbf{Steps to reproduce: }
	\begin{itemize}
		\item Click on the plus button on the dashboard while unknown Bluetooth devices are nearby
		\item Unpaired devices are listed and can be registered
	\end{itemize}
	\item \textbf{What is the current bug behavior? }It's possible to register unpaired devices.\\
	\item \textbf{What is the expected correct behavior? }It shouldn't be possible to register unpaired devices.\\
	\item \textbf{Reason? }Scan results are added to the list of devices in add device activity.\\
	\item \textbf{Possible fixes: }Do not add scan results to the list of devices in add device activity.\\
\end{itemize}



\item {[app] Background scan does not filter devices before reporting them (state = closed)}
\begin{itemize}	
	\item \textbf{Summary: }Background scan does not filter devices using the bloom filter before reporting them to the backend. Resulting in a lot of unnecessary traffic and power usage.\\
	\item \textbf{Steps to reproduce: }
	\begin{itemize}
		\item Start the app\\
		\item Have a few Bluetooth or BLE devices in the proximity.\\
	\end{itemize}
	\item \textbf{What is the current bug behavior? }Every spotted device is reported to the backend.\\
	\item \textbf{What is the expected correct behavior? }Only those devices listed in the BloomFilter provided by the backend should be reported to the backend.\\
	\item \textbf{Reason? }BloomFilter is not used (correctly).\\
	\item \textbf{Possible fixes: }Download and use BloomFilter.\\
\end{itemize}


\item {[app] User data is not cleared from app on logout (state = closed)}
\begin{itemize}	
	\item \textbf{Summary: }User data is not cleared from app on logout.\\
	\item \textbf{Steps to reproduce: }
	\begin{itemize}
		\item Log in\\
		\item Add a device\\
		\item Log out\\
		\item Log into a different account that does not have that device registered.\\
	\end{itemize}
	\item \textbf{What is the current bug behavior? }The added device is still there and is even permanently added to the second account through user data sync.\\
	\item \textbf{What is the expected correct behavior? }Only devices registered to the respective account should be displayed when switching accounts in the app.\\
	\item \textbf{Reason? }User data was not cleared from app on log out.\\
	\item \textbf{Possible fixes: }Clear user data from app on log out.\\
\end{itemize}



\item {[app] User data sync does not download updates to existing entries (state = closed)}
\begin{itemize}	
	\item \textbf{Summary: }User data sync does not download updates to existing entries.\\
	\item \textbf{Steps to reproduce: }
	\begin{itemize}
		\item Add a device on client A\\
		\item 	Restart the app on client A to trigger a sync\\
		\item 	(Re)start the app on client B to trigger a sync - make sure the device is correctly downloaded\\
		\item 	Modify the device on client A\\
		\item 	Restart the app on client A to trigger a sync\\
		\item 	(Re)start the app on client B to trigger a sync - the change is not applied\\
	\end{itemize}
	\item \textbf{What is the current bug behavior? }Changes on already synced devices are not synchronized to other clients.\\
	\item \textbf{What is the expected correct behavior? }Changes on already synced devices should get synchronized to other clients.\\
	\item \textbf{Reason? }Faulty implementation of UserDataSyncManager.\\
	\item \textbf{Possible fixes: }Fix implementation of UserDataSyncManager.\\
\end{itemize}



\item {[server-model] database changes are not committed to database (state = closed)}
\begin{itemize}	
	\item \textbf{Summary: }Database changes are not committed (=written back) to the database when using a properly injected EntityManager instance. This is probably a configuration problem.\\
	\item \textbf{Steps to reproduce: }
	\begin{itemize}
		\item Create a new Backoffice account and observe that despite the success message the new account does not exist.\\
	\end{itemize}
	\item \textbf{What is the current bug behavior? }When data is persisted to the database, the EntityManager signals success, but when reading from the database afterwards the changes are not there.\\
	\item \textbf{What is the expected correct behavior? }When data is persisted to the database, the EntityManager signals success and the changes are returned on subsequent queries.\\
	\item \textbf{Reason? }This is probably a configuration problem or an incompatibility between Hibernate and Glassfish.\\
	\item \textbf{Possible fixes: }Try migrating to a different JPA implementation as it might provide different diagnostics and potentially also different behaviour. Hibernate seems to be the preferred JPA implementation on JBoss, while Glassfish has EclipseLink preinstalled.\\
\end{itemize}

\end{enumerate}